% !TEX root = ../main.tex

%%%%%%%%%%%%%%%%%%%%%%
\chapter{Related Work}
\label{chap:related_work}
%%%%%%%%%%%%%%%%%%%%%%

The idea of allowing the fractional splitting of votes in Liquid Democracy is not entirely new. In 2014, Degrave first proposed "multi-proxy delegation", which closely resembles our understanding fractional delegation, in that each delegators can delegate to more than one proxy (delegate) at a time. They enforce in their implementation that the delegated vote is divided equally among the chosen proxies; for example, a voter delegating to three proxies would assign one third of their vote to each. Degrave also introduces a method for resolving multi-proxy delegations, which also entails constructing a system of linear equations, however being rather short, the article does not describe in detail how or why the method works, and does not evaluate any practical implementations of it. \cite{degraveResolvingMultiproxyTransitive2014}

Bersetche revisits and extends the idea of fractional delegation in 2022 under the term multi-agent delegation. Their approach allows an arbitrary fraction of votes to be delegated, not necessarily equal fractions to each delegate. Furthermore, voters are permitted to delegate part of their vote while still retaining a fraction for themselves—enabling them to vote directly and delegate simultaneously. They explore the "presence of equillibrium states" in "delegation games" using multi-agent delegation, meaning a collection of delegations so that no agent can unilaterally change their delegations to increase their voting power. The paper finds that such states exist for delegation graphs allowing multi-agent delegation.  \cite{bersetcheGeneralizingLiquidDemocracy2022}

Utke and Schmidt-Kraepelin use the term fractional delegation, although with a slightly different meaning than in this thesis. Their 2023 paper studies delegation rules that take as input ranked delegations, where each voter specifies a preference ordering over potential delegates, but not explicit vote fractions. The delegation rule then distributes voting power \textit{fractionally} across sinks based on these rankings. While delegators cannot directly assign fractions to each delegate, the outcome resembles fractional delegation, in that multiple sinks may receive fractional amounts of a single voter's power. The authors analyze such rules and show that, unlike non-fractional delegation rules, they can simultaneously satisfy desirable properties like anonymity, confluence, and copy-robustness. \cite{NEURIPS2023_dbb51809}

Nils Wandel has implemented fractional delegation and makes the product and its codebase available via a web interface and a GitHub repository. The website can be visited at \texttt{\url{electric.vote}}. No accompanying literature exists, but an inspection of the source code reveals that Wandel also resolves delegations using a system of linear equations. Unlike in this paper, Wandel does not calculate the power values of nodes explicitly, rather the code determines the outcome of votes with multiple option to vote for. The algorithm takes as input votes from sinks and delegations from delegators, and then first determines the standing power of each delegator, and then combines this with the votes of the sinks to determine the final aggregated result for each proposal. That this split of first resolving the standing power of delegators and then applying this to sinks is possible follows from the proof in section \cref{subsec:unique_sol}, where the matrix $P$ can be split into sub-matrices. Transposed, effectively turning it into the $W$ matrix of our resolution problem, $P$ looks as follows

\[
P^T = \begin{bmatrix}
Q & 0 \\
R & I_r
\end{bmatrix},
\]

The first rows of this matrix are the equations for the standing power of delegators (transient states). Evidently, as seen by the zero matrix next to $Q$, they don't depend on the standing power of any sinks, thus their standing power can be found out without knowing the standing power of any sinks. \cite{nilsElectricvote2020}
